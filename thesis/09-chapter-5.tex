\chapter{Conclusion} \label{chap:chap-5}
In summary, the novel technique introduced for encoding CV and DPV data represents a pivotal advancement in the realm of SDLs. By effectively segmenting voltammograms according to their distinct characteristics and showcasing its effectiveness across a spectrum of machine learning applications, from clustering and classification to denoising and synthetic data generation, this technique signifies a significant step in improving the automation of custom low-cost devices in SDLs. Machine learning models able to precisely encode chemical data from characterization results may be used to enhance high-throughput operations by integrating multiple low-cost devices using the same trained model. This approach streamlines the adoption of SDL and HT setups and facilitates their integration into diverse research endeavours.

Looking ahead, there is a vast landscape for further exploration, particularly in investigating alternative curve simplification algorithms and seamlessly integrating the encoding technique into operational SDL frameworks. This approach not only promises to significantly enhance the efficiency and accuracy of SDL setups but also holds the potential to revolutionize access to such technologies. By significantly reducing the entry barriers for new research groups interested in embarking on SDL and high-throughput setups, this advancement opens the doors to a more inclusive and collaborative scientific landscape, sparking new possibilities and inspiring future research.