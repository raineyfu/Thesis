\begin{center}
\chapter*{\centering Abstract} 
\begin{singlespace}
Self-driving laboratories (SDLs) represent a cutting-edge concept in scientific research and experimentation. SDLs utilize automated instruments, recommendation algorithms, and an orchestration device to conduct experiments and analyze data without human intervention. Among the array of experiments conducted by SDLs, cyclic voltammetry (CV) and differential pulse voltammetry (DPV) are prominent, offering insights into electrochemical processes. However, efficiently extracting crucial information, such as overall shape and peaks, from CV and DPV data remains challenging. This thesis presents a novel encoding technique tailored for CV and DPV data to enhance SDLs\textquotesingle{} understanding of chemical environments. With this encoding method, SDLs can discern intricate patterns and relationships within the data more effectively. Experiments consisting of various machine learning tasks, such as clustering, classification, denoising, and synthetic data generation, that an SDL may encounter showed excellent results. Beyond SDLs, the utility of this encoding technique extends to any 2-dimensional data. Its versatility opens avenues for broader scientific and industrial applications, empowering researchers and practitioners to glean valuable insights from complex datasets. As SDLs continue to evolve, incorporating innovative methodologies such as this encoding technique promises to accelerate scientific discovery and advance technological frontiers.
\end{singlespace}
\end{center}