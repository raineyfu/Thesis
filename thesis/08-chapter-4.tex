\chapter{Denoising} \label{chap:chap-4}

\section{Introduction}
Previous works have developed low-cost potentiostat alternatives that return competitive results compared to the traditional platform based on commercial options \cite{PabloGarca2024}. However, these results could still be improved. To try and improve these results, data from commercial options is used for comparison. 
\section{AutoEncoder}
An autoencoder is a neural network used to learn an efficient low-dimensional encoding of data. An autoencoder consists of an encoder and decoder. The encoder transforms the input data into an encoded representation, and the decoder attempts to recreate the data from the encoded representation. Since the goal is to try and improve the data quality, the commercial potentiostat data is used for the decoder instead. This way, the low-cost potentiostat data is used to create an encoded representation, and an  equivalent commercial potentiostat data is decoded. The main problem to solve is how to pair results from the two potentiostats. The metal and ligand used for each experiment are recorded. However, there are many other variables that can impact the data. As such, the clustering technique described previously to pair similar experimental results that use the same metal and ligand. 
\section{Results and Discussion}

\begin{figure}[h!]
  \centering
    \includegraphics[width=1\textwidth]{figures/autoencoder.png}
    \caption{AutoEncoder Results}
    \label{autoncoder_results}
\end{figure}
In Figure \ref{autoncoder_results}, both the
As seen in Figure \ref{autoncoder_results}, both the input and output are similar in overall shape. However, the output contains a much more defined duck-shaped voltammogram, which is typically expected. The results show promising outcomes and indicate that an autoencoder can be effectively transform data from the low-cost potentiostat to resemble data from the commercial potentiostat. By leveraging the capacity of deep neural networks to learn complex patterns and relationships within the data, it becomes feasible to enhance the quality of measurements obtained from low-cost instruments, thereby expanding their utility in research and industrial applications.
However, despite the promising results, several drawbacks and considerations must be acknowledged. Firstly, the effectiveness of the transformation heavily relies on the quality and diversity of the training data. Insufficient or biased training samples may lead to suboptimal performance and generalization issues, especially when dealing with complex electrochemical processes or diverse experimental conditions. While the autoencoder can effectively capture and replicate the dominant features present in the data, it may struggle with preserving subtle nuances or domain-specific characteristics inherent to the commercial potentiostat. Variations in hardware specifics, measurement protocols, or environmental factors could introduce discrepancies between the transformed and reference datasets. 
In conclusion, while autoencoders offer a promising avenue for enhancing the capabilities of low-cost potentiostats, their deployment must be accompanied by rigorous validation and consideration of the aforementioned limitations. Future research could focus on optimizing the autoencoder architecture, exploring alternative deep learning techniques, and investigating strategies for addressing data heterogeneity to further improve the robustness and versatility of the proposed approach.
