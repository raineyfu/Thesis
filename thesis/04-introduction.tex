\chapter{Introduction} \label{introduction}
Amidst urgent global challenges like climate change, energy sustainability, and healthcare crises, there is a growing need for efficient solutions to address the demands of a growing population and increasing resource demands. Accelerating advancements in materials, technology, and scientific knowledge offer potential avenues for tackling these challenges. However, conventional research methods, marked by gradual progress and limited efficiency, may fall short of meeting the urgency posed by these issues. Self-driving laboratories (SDLs), which integrate laboratory automation and data-driven decision-making, emerge as promising tools to expedite and streamline the exploration of solutions while presenting several advantages over traditional scientific approaches \cite{Tom2024}. Developing a fully autonomous self-driving laboratory is a complex endeavor that combines various research disciplines. Machine learning and modeling techniques are utilized to forecast materials properties and propose new experiments. SDLs typically use Bayesian optimization to guide their decision-making algorithm. An example of this is Atlas, a brain for SDLs that has been used to identify the voltage peak in CV experiments to optimize the oxidation potential of a set of metal complexes \cite{Hickman2023}. Concurrently, robotics, computer vision, and automated characterization methods are employed to conduct experiments and analyze outcomes. Central to the design of autonomous labs is the integration of these disparate technologies into a cohesive platform, facilitating seamless interaction between experiments and computational modeling \cite{StriethKalthoff2023}. 

SDLs can conduct experiments autonomously, performing tasks with increased speed and precision compared to manual processes. Moreover, they utilize data-driven algorithms to navigate through experimental spaces, enabling efficient exploration based on feedback from existing data, a process known as "closed-loop" experimentation. Additionally, SDLs address issues such as reproducibility challenges and the underrepresentation of negative results in scientific literature by promoting the digitization of research processes. Through automated systems, experimental protocols are meticulously documented, enhancing repeatability and reproducibility. Furthermore, digitization facilitates comprehensive data recording and sharing, emphasizing the importance of negative or null results, thus providing a more accurate depiction of scientific endeavors. The wealth of high-quality data generated by autonomous experimentation serves as a valuable resource for the development of artificial intelligence (AI) in materials science and chemistry. By improving machine learning (ML) and deep learning (DL) models, this data enhances the decision-making capabilities of SDLs, furthering their effectiveness in optimizing materials or processes and facilitating novel discoveries.

SDLs in chemistry and materials science are characterized by two critical dimensions: software autonomy and hardware autonomy. Regarding software autonomy, which governs experiment selection, SDLs are categorized into three types: (1) single iterations of automated experimentation with data-driven methods for selecting subsequent experiments, (2) multiple iterations within closed-loop systems where experimental results inform subsequent rounds of automated experiments, and (3) generative approaches involving multiple iterations of closed-loop optimization within algorithmically generated search or chemical spaces. By automating high-throughput experimentation and streamlining experiment planning and execution, SDLs possess the potential to substantially accelerate research in chemistry and materials discovery. SDLs have played a pivotal role and made noteworthy advancements in various fields including drug discovery, genomics, chemistry, and materials science \cite{Tom2024}.