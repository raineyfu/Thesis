\chapter{Background and Motivation} \label{chap:chap-1}

% if you want a short header you can use the following command
% \chapter[short-header-name]{chapter-title} \label{chap:chap-1}


% add your chapter text here
\section{Electrochemistry}
Electrochemistry is the branch of chemistry studying electron mobility, which leads to the phenomenon known as electricity. This flow of electrons occurs through the transfer from one chemical species to another in what is called an oxidation-reduction reaction. When a substance loses an electron, its oxidation state increases, indicating oxidation. When a substance acquires an electron, its oxidation state decreases, indicating reduction. For example, consider the following redox reaction which has oxidation and reduction components:\\
\begin{align}
H\textsubscript{2} + F\textsubscript{2} \to 2HF
\end{align}\\
Oxidation:\\
\begin{align}
H\textsubscript{2} \to 2H\textsuperscript{+} + 2e\textsuperscript{-}
\end{align}\\
Reduction: \\
\begin{align}
F\textsubscript{2} + 2e\textsuperscript{-} \to 2F\textsuperscript{-}
\end{align}\\
An electrode serves as a stable electrical conductor, facilitating the flow of electrical current within non-metallic solids, liquids, gases, plasmas, or even vacuums. While electrodes often exhibit high electrical conductivity, they are not limited to metals \cite{Faraday_1970}. An electrochemical cell is a device capable of either producing electrical energy through internal chemical reactions or utilizing supplied electrical energy to drive chemical processes within it. This device effectively transforms chemical energy into electrical energy or vice-versa. In an electrochemical cell, reduction and oxidation reactions take place at the electrodes. The electrode where reduction occurs is termed the cathode, while oxidation occurs at the anode.\\
Electrode potential is the voltage of an electrochemical cell composed of a reference electrode and another electrode to be characterized \cite{goldbook}

\section{Cyclic Voltammetry}

Cyclic voltammetry is a common electrochemical technique that generates important reduction and oxidation information about different molecules \cite{doi:10.1021/ac60210a007}. Typically, the working electrode potential increases linearly with time. After a set potential is reached, the potential decreases to return to the initial potential. Theses cycles can be repeated as many times as needed. The rate of voltage change over time is known as the experiment's scan rate (V/s) \cite{https://doi.org/10.1002/anie.198408313}. Cyclic voltammetry serves as a valuable tool for studying qualitative information about electrochemical processes across diverse conditions. It enables the examination of intermediates in oxidation-reduction reactions and the assessment of reaction reversibility. Moreover, CV facilitates the determination of electron stoichiometry, analyte diffusion coefficients, and formal reduction potentials, aiding in identification processes. Additionally, in reversible, Nernstian systems, the proportional relationship between concentration and current allows for the determination of unknown solution concentrations via the construction of calibration curves correlating current and concentration \cite{Libretexts_2023}.
In cyclic voltammetry, peaks represent electrochemical processes occurring at the electrode surface. The anodic peak ($E_{p, a}$) is observed during the scan where oxidation of the electroactive species occurs at the electrode and corresponds to the potential at which oxidation is most favourable. The current increases as the potential applied to the electrode becomes more positive, reaching a maximum at the peak potential. The cathodic peak is observed during the reverse scan where reduction of the electroactive species occurs at the working electrode and corresponds to the potential at which reduction is most favorable. The current increases as the potential becomes more negative, reaching a maximum at the peak potential \cite{GRIMSHAW20001}. Typically, researchers are especially interested in these peaks.
\section{Differential Pulse Voltammetry}
Differential Pulse Voltammetry (DPV) is an electrochemical measurement technique from linear sweep voltammetry \cite{Scholz2005-pa}. The current is measured right before each potential alteration, and the difference in current is plotted against the potential. This method helps reduce the impact of charging current by sampling the current just before the potential change. DPV is well suited for measurements with extremely low concentrations of chemicals. This is because the effect of the charging current can be minimized to achieve high sensitivity, and only the faradaic current, the electric current generated by the redox of a chemical at an electrode, is extracted, so electrode reactions can be measured precisely. 

\section{Potentiostat}
A potentiostat is an electronic device used to control the working electrode's potential in a multiple electrode electrochemical cell \cite{J2002-fn}. Most labs use potentiostats provided by commercial vendors, which are typically governed by proprietary software, employ graphical user interfaces (GUI), and produce processed data. These potentiostats lack the capability for comprehensive control using an application programming interface (API) and direct access to unprocessed measurements. While convenient for manual tasks, these characteristics present difficulties when integrating into automated systems, highlighting the need for potentiostats that are thoroughly digitized to facilitate data-rich experiments and electrochemical process analysis in modern self-driving laboratories. The Matter Lab has developed an open-source potentiostat along with open-source firmware and interface \cite{PabloGarca2024}. Notably, the instrument is affordable and compact, making it particularly advantageous for groups with budget constraints or those establishing their initial self-driving laboratory.  

When working with a potentiostat, the working electrodes should be immediately polished after use to ensure there are no surface contaminants that inhibit electron transfer. Even a few hours of air exposure will degrade the electrode surface. 

\section{The Matter Lab}
The Matter Lab is a research group at the University of Toronto. One of the main research areas is materials discovery with self-driving synthetic laboratory. Developing a fully autonomous self-driving laboratory is a complex endeavor that combines various research disciplines. Machine learning and modeling techniques are utilized to forecast materials properties and propose new experiments. Concurrently, robotics, computer vision, and automated characterization methods are employed to conduct experiments and analyze outcomes. Central to the design of autonomous labs is the integration of these disparate technologies into a cohesive platform, facilitating seamless interaction between experiments and computational modeling \cite{StriethKalthoff2023}.
Within the self-driving laboratory (SDL) subgroup, exploration spans multiple domains, encompassing artificial intelligence and optimization methods for experiment control and design, robotics systems for execution, and automatic characterization methods for result analysis. A novel research avenue involves leveraging computer vision to develop visually-aware robotic systems capable of executing chemical and materials science experiments.
By automating high-throughput experimentation and streamlining experiment planning and execution, SDLs possess the potential to substantially accelerate research in chemistry and materials discovery. SDLs have played a pivotal role and made noteworthy advancements in various fields including drug discovery, genomics, chemistry, and materials science. 
SDLs typically use Bayesian optimization to guide its decision-making algorithm. Atlas, a brain for SDLs used software to identify the voltage peak in CV experiments to optimize the oxidation potential of a set of metal complexes. 